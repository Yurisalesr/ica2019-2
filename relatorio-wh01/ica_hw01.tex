\documentclass[conference]{IEEEtran}
\IEEEoverridecommandlockouts
% The preceding line is only needed to identify funding in the first footnote. If that is unneeded, please comment it out.
\usepackage[utf8]{inputenc}   % Caracteres
\usepackage[brazil]{babel}    % Língua portuguesa
\usepackage{cite}
\usepackage{amsmath,amssymb,amsfonts}
\usepackage{algorithmic}
\usepackage{graphicx}
\usepackage{textcomp}
\usepackage{xcolor}
\usepackage{acro} % Acrônimos
\DeclareAcronym{eda}{
    short = EDA,
    long = análise exploratória de dados,
    foreign = \textit{exploratory data analysis}
}

\DeclareAcronym{pca}{
    short = PCA,
    long = análise das componentes principais,
    foreign = \textit{principal component analysis}
}

\DeclareAcronym{qoe}{
    short = QoE,
    long = qualidade de experiência,
    foreign = \textit{quality of experience}
}

\DeclareAcronym{mos}{
    short = MOS,
    long = pontuação de opinião média,
    foreign = \textit{mean opinion score}
}

% %%%%%%%%%%%%%%%%%%%%%%%%%%%%%%%%%%%%%%%%%%%%%%%%%%%%

% \DeclareAcronym{5G}{
%  short = 5G,
%  long = quinta geração,
%  foreign = \textit{fifth generation}
% }

% \DeclareAcronym{LOS}{
%  short = LOS,
%  long = linha de visada,
%  foreign = \textit{line-of-sight}
% }

% \DeclareAcronym{MIMO}{
%  short = MIMO,
%  long = múltiplas entradas - múltiplas saídas,
%  foreign = \textit{multiple-input multiple-output}
% }

% \DeclareAcronym{BF}{
%  short = BF,
%  long = formatação de feixes,
%  foreign = \textit{beamforming}
% }

% \DeclareAcronym{DBF}{
%  short = DBF,
%  long = formatação de feixes distribuída,
%  foreign = \textit{distributed beamforming}
% }

% \DeclareAcronym{BS}{
%  short = BS,
%  long = estação base,
%  foreign = \textit{base station}
% }

% \DeclareAcronym{UE}{
%  short= UE,
%  long = terminal móvel,
%  foreign = \textit{user equipment}
% }

% \DeclareAcronym{AWGN}{
%  short= AWGN,
%  long = aditivo Gaussiano branco,
%  foreign = \textit{additive white Gaussian noise}
% }

% \DeclareAcronym{iid}{
%     short = \textit{i.i.d.},
%     long = independentes e identicamente distribuídos
% }

% \DeclareAcronym{mmw}{
%  short= mmWave,
%  long = ondas milimétricas,
%  foreign = \textit{millimeter-wave}
% }

% \DeclareAcronym{SINR}{
%   short = SINR,
%   long = relação sinal-ruído mais interferência,  
%   foreign = \textit{signal-to-interference-plus-noise ratio}
% }

% \DeclareAcronym{SNR}{
%   short = SNR,
%   long =  relação sinal-ruído,
%   foreign = \textit{signal-to-noise ratio}
% }

% \DeclareAcronym{MMSE}{
%     short = MMSE,
%     long = mínimo erro quadrádico médio,
%     foreign = \textit{minimum mean square error}
% }

% \DeclareAcronym{AW-MMSE}{
%     short = AW-MMSE,
%     long = mínimo erro quadrádico médio adaptativamente ponderado,
%     foreign = \textit{adaptive weighted minimum mean square error}
% }

% \DeclareAcronym{ULA}{
%     short = ULA,
%     long = arranjo de antenas linear uniforme,
%     foreign = \textit{uniform linear array}
% }

% \DeclareAcronym{UPA}{
%     short = UPA,
%     long = arranjo planar uniforme,
%     foreign = \textit{uniform planar array}
% }

%\chapter*{Acronyms}
%\renewcommand{\titulonome}{Acronyms}%
%\renewcommand{\prepbynome}{UFC.33 Team}%

%\begin{singlespace}
% \begin{acronym}%[LTE-Advanced]%\addtolength{\itemsep}{-0.5\baselineskip}
%   \acro{2G}{Second Generation}
%   \acro{3G}{3$^\text{rd}$~Generation}
%   \acro{3GPP}{3$^\text{rd}$~generation partnership project}
%   \acro{4G}{4$^\text{th}$~Generation}
%   \acro{5G}{5$^\text{th}$~generation}
%   \acro{AE}{Associate Editor}
%   \acro{AWGN}{additive white Gaussian noise}
%   \acro{BER}{bit error rate}
%   \acro{BF}{beamforming}
%   \acro{BLER}{BLock Error Rate}
%   \acro{BPC}{Binary Power Control}
%   \acro{BPSK}{Binary Phase-Shift Keying}
%   \acro{BRA}{Balanced Random Allocation}
%   \acro{BS}{base station}
%   \acro{CAP}{Combinatorial Allocation Problem}
%   \acro{CAPEX}{Capital Expenditure}
%   \acro{CBF}{coordinated Beamforming}
%   \acro{CDF}{cumulative distribution function}
%   \acro{CL}{closed loop}
%   \acro{CS}{Coordinated Scheduling}
%   \acro{CSI}{channel state information}
%   \acro{CSIT}{channel state information at the transmitter}
%   \acro{CDI}{channel distribution information}
%   \acro{D2D}{device-to-device}
%   \acro{DCA}{Dynamic Channel Allocation}
%   \acro{DE}{Differential Evolution}
%   \acro{DFT}{Discrete Fourier Transform}
%   \acro{DIST}{Distance}
%   \acro{DL}{downlink}
%   \acro{DLPR}{downlink Power Reduction}
%   \acro{DMA}{Double Moving Average}
%   \acro{DMRS}{Demodulation Reference Signal}
%   \acro{D2DM}{D2D Mode}
%   \acro{DMS}{D2D Mode Selection}
%   \acro{DPC}{Dirty Paper Coding}
%   \acro{DRA}{Dynamic Resource Assignment}
%   \acro{DSA}{Dynamic Spectrum Access}
%   \acro{DTDD}{dynamic time division duplex}
%   \acro{EE}{energy efficiency}
%   \acro{eIMTA}{enhanced	interference mitigation and traffic adaptation}
%   \acro{eNB}{evolved Node B}
%   \acro{FDD}{frequency division duplex}
%   \acro{HD}{half duplex}
%   \acro{IA}{interference aligment}
%   \acro{IC}{interference channel}
%   \acro{IPNP}{interference plus noise power}
%   \acro{LoS}{line-of-sight}
%   \acro{LTE}{Long-term evolution}
%   \acro{METIS}{Mobile Enablers for the Twenty-Twenty Information Society}
%   \acro{MIMO}{multiple-input multiple-output}
%   \acro{MIMO-IC}{MIMO-interference channel}
%   \acro{MISO}{multiple-input single-output}
%   \acro{MISO-IC}{MISO-interference channel}
%   \acro{MRC}{maximum ratio combining}
%   \acro{MRT}{maximum ratio transmission}
%   \acro{MS}{mode selection}
%   \acro{MSE}{mean square error}
%   \acro{MMSE}{minimum mean square error}
%   \acro{MTC}{machine type communications}
%   \acro{NLoS}{non-line-of-sight}
%   \acro{NSPS}{national security and public safety}
%   \acro{NWC}{network coding}
%   \acro{PB}{pricing-based}
%   \acro{PBA}{pricing-based algorithm}
%   \acro{PC}{power control}
%   \acro{PHY}{physical layer}
%   \acro{PRB}{Physical Resource Block}
%   \acro{QCQP}{quadratically constrained quadratic program}
%   \acro{QoS}{Quality of Service}
%   \acro{QPSK}{Quadri-Phase Shift Keying}
%   \acro{RAISES}{Reallocation-based Assignment for Improved Spectral Efficiency and Satisfaction}
%   \acro{RAN}{Radio Access Network}
%   \acro{RA}{Resource Allocation}
%   \acro{RAT}{Radio Access Technology}
%   \acro{RB}{resource block}
%   \acro{RF}{radio frequency}
%   \acro{SDR}{semidefinite relaxation}
%   \acro{SINR}{signal-to-interference-plus-noise ratio}
%   \acro{SISO}{single-input single-output}
%   \acro{SOCP}{second order cone programming}
%   \acro{SPBA}{sub-optimal pricing-based algorithm}
%   \acro{SNR}{signal-to-noise ratio}
%   \acro{STC}{space-time coding}
%   \acro{TDD}{time division duplexing}
%   \acro{TNFD}{three node full duplex}
%   \acro{TTI}{Transmission Time Interval}
%   \acro{UE}{user equipment}
%   \acro{UL}{uplink}
%   \acro{ULPB}{uplink Power Boosting}
%   \acro{VUE}{vehicular user equipment}
%   \acro{V2X}{vehicle-to-vehicle and vehicle-to-infrastructure}
%   \acro{ZF}{zero-forcing}
%   \acro{ZMCSCG}{Zero Mean Circularly Symmetric Complex Gaussian}
% \end{acronym}
%\end{singlespace}  % Acrônimos
\def\BibTeX{{\rm B\kern-.05em{\sc i\kern-.025em b}\kern-.08em
    T\kern-.1667em\lower.7ex\hbox{E}\kern-.125emX}}
\begin{document}

\title{Processamento de Dados para Predição do Nível de QoE de %Qualidade de Experiência
Usuários em uma Rede de Telefonia Móvel\\
% {\footnotesize \textsuperscript{*}Note: Sub-titles are not captured in Xplore and
% should not be used}
% \thanks{Identify applicable funding agency here. If none, delete this.}
}

\author{
Ingrid S. M. Furtado,
Ezequias M. S. de Santana Jr.,
Yuri S. Ribeiro
% \thanks{Este trabalho foi parcialmente financiado pelo CNPq.} 
\\ \vspace{0.1cm}
% \small Grupo de Pesquisa em Telecomunicações sem Fio (GTEL), Universidade Federal do Cear\'a, Fortaleza, Brasil.\\
\small Engenharia de Telecomunicações\\ 
\small Departamento de Engenharia de Teleinformática (DETI)\\ 
\small Universidade Federal do Cear\'a, Fortaleza, Brasil.\\
% \small E-mails: \texttt{\{ingrid, ezequias, yuri\}@gtel.ufc.br}
}

% \author{\IEEEauthorblockN{ Given Name Surname}
% % \IEEEauthorblockA{\textit{dept. name of organization (of Aff.)} \\
% % \textit{name of organization (of Aff.)}\\
% % City, Country \\
% % email address}
% \and
% \IEEEauthorblockN{ Given Name Surname}
% % \IEEEauthorblockA{\textit{dept. name of organization (of Aff.)} \\
% % \textit{name of organization (of Aff.)}\\
% % City, Country \\
% % email address}
% \and
% \IEEEauthorblockN{ Given Name Surname}
% % \IEEEauthorblockA{\textit{dept. name of organization (of Aff.)} \\
% % \textit{name of organization (of Aff.)}\\
% % City, Country \\
% % email address}
% }

\maketitle

\begin{abstract}
blabla

\end{abstract}

\begin{IEEEkeywords}
% 5
pré-processamento, análise exploratória de dados, qualidade de experiência, 
\textit{mean opinion score}, redes móveis.
\end{IEEEkeywords}

\section{Introdução}
% input é um comando ideal para esse tipo de operação 
intro \cite{h2020}
\cite{islR}
\ac{mos} 

\section{Metodologia}
Texto explicativo/introdutório antes das subsec...
\subsection{O Conjunto de Dados}
falar do data set

\subsection{Pré-Processamento dos Dados}
o que foi feito e no como foi feito falar da teoria..

\subsection{Análise Exploratória dos Dados}
\ac{eda}

foi feito...Análise univariada incondicional dos dados em seguida.

Análise univariada condicionada dos dados

Análise bivariada incondicional

Análise multivariada incondicional discutida na próxima subseção

\subsection{Análise das Componentes Principais}
\ac{pca}

\section{Resultados}
\input{text/resultados}
% \section{Ease of Use}

% \subsection{Maintaining the Integrity of the Specifications}

% The IEEEtran class file is used to format your paper and style the text. All margins, 
% column widths, line spaces, and text fonts are prescribed; please do not 
% alter them. You may note peculiarities. For example, the head margin
% measures proportionately more than is customary. This measurement 
% and others are deliberate, using specifications that anticipate your paper 
% as one part of the entire proceedings, and not as an independent document. 
% Please do not revise any of the current designations.

% \section{Prepare Your Paper Before Styling}
% Before you begin to format your paper, first write and save the content as a 
% separate text file. Complete all content and organizational editing before 
% formatting. Please note sections \ref{AA}--\ref{SCM} below for more information on 
% proofreading, spelling and grammar.

% Keep your text and graphic files separate until after the text has been 
% formatted and styled. Do not number text heads---{\LaTeX} will do that 
% for you.

% \subsection{Abbreviations and Acronyms}\label{AA}
% Define abbreviations and acronyms the first time they are used in the text, 
% even after they have been defined in the abstract. Abbreviations such as 
% IEEE, SI, MKS, CGS, ac, dc, and rms do not have to be defined. Do not use 
% abbreviations in the title or heads unless they are unavoidable.

% \subsection{Units}
% \begin{itemize}
% \item Use either SI (MKS) or CGS as primary units. (SI units are encouraged.) English units may be used as secondary units (in parentheses). An exception would be the use of English units as identifiers in trade, such as ``3.5-inch disk drive''.
% \item Avoid combining SI and CGS units, such as current in amperes and magnetic field in oersteds. This often leads to confusion because equations do not balance dimensionally. If you must use mixed units, clearly state the units for each quantity that you use in an equation.
% \item Do not mix complete spellings and abbreviations of units: ``Wb/m\textsuperscript{2}'' or ``webers per square meter'', not ``webers/m\textsuperscript{2}''. Spell out units when they appear in text: ``. . . a few henries'', not ``. . . a few H''.
% \item Use a zero before decimal points: ``0.25'', not ``.25''. Use ``cm\textsuperscript{3}'', not ``cc''.)
% \end{itemize}

% \subsection{Equations}
% Number equations consecutively. To make your 
% equations more compact, you may use the solidus (~/~), the exp function, or 
% appropriate exponents. Italicize Roman symbols for quantities and variables, 
% but not Greek symbols. Use a long dash rather than a hyphen for a minus 
% sign. Punctuate equations with commas or periods when they are part of a 
% sentence, as in:
% \begin{equation}
% a+b=\gamma\label{eq}
% \end{equation}

% Be sure that the 
% symbols in your equation have been defined before or immediately following 
% the equation. Use ``\eqref{eq}'', not ``Eq.~\eqref{eq}'' or ``equation \eqref{eq}'', except at 
% the beginning of a sentence: ``Equation \eqref{eq} is . . .''

% \subsection{\LaTeX-Specific Advice}

% Please use ``soft'' (e.g., \verb|\eqref{Eq}|) cross references instead
% of ``hard'' references (e.g., \verb|(1)|). That will make it possible
% to combine sections, add equations, or change the order of figures or
% citations without having to go through the file line by line.

% Please don't use the \verb|{eqnarray}| equation environment. Use
% \verb|{align}| or \verb|{IEEEeqnarray}| instead. The \verb|{eqnarray}|
% environment leaves unsightly spaces around relation symbols.

% Please note that the \verb|{subequations}| environment in {\LaTeX}
% will increment the main equation counter even when there are no
% equation numbers displayed. If you forget that, you might write an
% article in which the equation numbers skip from (17) to (20), causing
% the copy editors to wonder if you've discovered a new method of
% counting.

% {\BibTeX} does not work by magic. It doesn't get the bibliographic
% data from thin air but from .bib files. If you use {\BibTeX} to produce a
% bibliography you must send the .bib files. 

% {\LaTeX} can't read your mind. If you assign the same label to a
% subsubsection and a table, you might find that Table I has been cross
% referenced as Table IV-B3. 

% {\LaTeX} does not have precognitive abilities. If you put a
% \verb|\label| command before the command that updates the counter it's
% supposed to be using, the label will pick up the last counter to be
% cross referenced instead. In particular, a \verb|\label| command
% should not go before the caption of a figure or a table.

% Do not use \verb|\nonumber| inside the \verb|{array}| environment. It
% will not stop equation numbers inside \verb|{array}| (there won't be
% any anyway) and it might stop a wanted equation number in the
% surrounding equation.

% \subsection{Some Common Mistakes}\label{SCM}
% \begin{itemize}
% \item The word ``data'' is plural, not singular.
% \item The subscript for the permeability of vacuum $\mu_{0}$, and other common scientific constants, is zero with subscript formatting, not a lowercase letter ``o''.
% \item In American English, commas, semicolons, periods, question and exclamation marks are located within quotation marks only when a complete thought or name is cited, such as a title or full quotation. When quotation marks are used, instead of a bold or italic typeface, to highlight a word or phrase, punctuation should appear outside of the quotation marks. A parenthetical phrase or statement at the end of a sentence is punctuated outside of the closing parenthesis (like this). (A parenthetical sentence is punctuated within the parentheses.)
% \item A graph within a graph is an ``inset'', not an ``insert''. The word alternatively is preferred to the word ``alternately'' (unless you really mean something that alternates).
% \item Do not use the word ``essentially'' to mean ``approximately'' or ``effectively''.
% \item In your paper title, if the words ``that uses'' can accurately replace the word ``using'', capitalize the ``u''; if not, keep using lower-cased.
% \item Be aware of the different meanings of the homophones ``affect'' and ``effect'', ``complement'' and ``compliment'', ``discreet'' and ``discrete'', ``principal'' and ``principle''.
% \item Do not confuse ``imply'' and ``infer''.
% \item The prefix ``non'' is not a word; it should be joined to the word it modifies, usually without a hyphen.
% \item There is no period after the ``et'' in the Latin abbreviation ``et al.''.
% \item The abbreviation ``i.e.'' means ``that is'', and the abbreviation ``e.g.'' means ``for example''.
% \end{itemize}
% An excellent style manual for science writers is \cite{b7}.

% \subsection{Authors and Affiliations}
% \textbf{The class file is designed for, but not limited to, six authors.} A 
% minimum of one author is required for all conference articles. Author names 
% should be listed starting from left to right and then moving down to the 
% next line. This is the author sequence that will be used in future citations 
% and by indexing services. Names should not be listed in columns nor group by 
% affiliation. Please keep your affiliations as succinct as possible (for 
% example, do not differentiate among departments of the same organization).

% \subsection{Identify the Headings}
% Headings, or heads, are organizational devices that guide the reader through 
% your paper. There are two types: component heads and text heads.

% Component heads identify the different components of your paper and are not 
% topically subordinate to each other. Examples include Acknowledgments and 
% References and, for these, the correct style to use is ``Heading 5''. Use 
% ``figure caption'' for your Figure captions, and ``table head'' for your 
% table title. Run-in heads, such as ``Abstract'', will require you to apply a 
% style (in this case, italic) in addition to the style provided by the drop 
% down menu to differentiate the head from the text.

% Text heads organize the topics on a relational, hierarchical basis. For 
% example, the paper title is the primary text head because all subsequent 
% material relates and elaborates on this one topic. If there are two or more 
% sub-topics, the next level head (uppercase Roman numerals) should be used 
% and, conversely, if there are not at least two sub-topics, then no subheads 
% should be introduced.

% \subsection{Figures and Tables}
% \paragraph{Positioning Figures and Tables} Place figures and tables at the top and 
% bottom of columns. Avoid placing them in the middle of columns. Large 
% figures and tables may span across both columns. Figure captions should be 
% below the figures; table heads should appear above the tables. Insert 
% figures and tables after they are cited in the text. Use the abbreviation 
% ``Fig.~\ref{fig}'', even at the beginning of a sentence.

% \begin{table}[htbp]
% \caption{Table Type Styles}
% \begin{center}
% \begin{tabular}{|c|c|c|c|}
% \hline
% \textbf{Table}&\multicolumn{3}{|c|}{\textbf{Table Column Head}} \\
% \cline{2-4} 
% \textbf{Head} & \textbf{\textit{Table column subhead}}& \textbf{\textit{Subhead}}& \textbf{\textit{Subhead}} \\
% \hline
% copy& More table copy$^{\mathrm{a}}$& &  \\
% \hline
% \multicolumn{4}{l}{$^{\mathrm{a}}$Sample of a Table footnote.}
% \end{tabular}
% \label{tab1}
% \end{center}
% \end{table}

% \begin{figure}[htbp]
% %\centerline{\includegraphics{fig1.png}}
% \caption{Example of a figure caption.}
% \label{fig}
% \end{figure}

% Figure Labels: Use 8 point Times New Roman for Figure labels. Use words 
% rather than symbols or abbreviations when writing Figure axis labels to 
% avoid confusing the reader. As an example, write the quantity 
% ``Magnetization'', or ``Magnetization, M'', not just ``M''. If including 
% units in the label, present them within parentheses. Do not label axes only 
% with units. In the example, write ``Magnetization (A/m)'' or ``Magnetization 
% \{A[m(1)]\}'', not just ``A/m''. Do not label axes with a ratio of 
% quantities and units. For example, write ``Temperature (K)'', not 
% ``Temperature/K''.

% \section*{Acknowledgment}

% The preferred spelling of the word ``acknowledgment'' in America is without 
% an ``e'' after the ``g''. Avoid the stilted expression ``one of us (R. B. 
% G.) thanks $\ldots$''. Instead, try ``R. B. G. thanks$\ldots$''. Put sponsor 
% acknowledgments in the unnumbered footnote on the first page.

% \section*{Referências}

% Please number citations consecutively within brackets \cite{b1}. The 
% sentence punctuation follows the bracket \cite{b2}. Refer simply to the reference 
% number, as in \cite{b3}---do not use ``Ref. \cite{b3}'' or ``reference \cite{b3}'' except at 
% the beginning of a sentence: ``Reference \cite{b3} was the first $\ldots$''

% Number footnotes separately in superscripts. Place the actual footnote at 
% the bottom of the column in which it was cited. Do not put footnotes in the 
% abstract or reference list. Use letters for table footnotes.

% Unless there are six authors or more give all authors' names; do not use 
% ``et al.''. Papers that have not been published, even if they have been 
% submitted for publication, should be cited as ``unpublished'' \cite{b4}. Papers 
% that have been accepted for publication should be cited as ``in press'' \cite{b5}. 
% Capitalize only the first word in a paper title, except for proper nouns and 
% element symbols.

% For papers published in translation journals, please give the English 
% citation first, followed by the original foreign-language citation \cite{b6}.

% \begin{thebibliography}{00}
% \bibitem{b1} G. Eason, B. Noble, and I. N. Sneddon, ``On certain integrals of Lipschitz-Hankel type involving products of Bessel functions,'' Phil. Trans. Roy. Soc. London, vol. A247, pp. 529--551, April 1955.
% \bibitem{b2} J. Clerk Maxwell, A Treatise on Electricity and Magnetism, 3rd ed., vol. 2. Oxford: Clarendon, 1892, pp.68--73.
% \bibitem{b3} I. S. Jacobs and C. P. Bean, ``Fine particles, thin films and exchange anisotropy,'' in Magnetism, vol. III, G. T. Rado and H. Suhl, Eds. New York: Academic, 1963, pp. 271--350.
% \bibitem{b4} K. Elissa, ``Title of paper if known,'' unpublished.
% \bibitem{b5} R. Nicole, ``Title of paper with only first word capitalized,'' J. Name Stand. Abbrev., in press.
% \bibitem{b6} Y. Yorozu, M. Hirano, K. Oka, and Y. Tagawa, ``Electron spectroscopy studies on magneto-optical media and plastic substrate interface,'' IEEE Transl. J. Magn. Japan, vol. 2, pp. 740--741, August 1987 [Digests 9th Annual Conf. Magnetics Japan, p. 301, 1982].
% \bibitem{b7} M. Young, The Technical Writer's Handbook. Mill Valley, CA: University Science, 1989.
% \end{thebibliography}
% \vspace{12pt}
% \color{red}
% IEEE conference templates contain guidance text for composing and formatting conference papers. Please ensure that all template text is removed from your conference paper prior to submission to the conference. Failure to remove the template text from your paper may result in your paper not being published.

% Arquivo de referências
\bibliographystyle{IEEEtran}
\bibliography{refs}

\end{document}
